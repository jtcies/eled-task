\documentclass[]{article}
\usepackage{lmodern}
\usepackage{amssymb,amsmath}
\usepackage{ifxetex,ifluatex}
\usepackage{fixltx2e} % provides \textsubscript
\ifnum 0\ifxetex 1\fi\ifluatex 1\fi=0 % if pdftex
  \usepackage[T1]{fontenc}
  \usepackage[utf8]{inputenc}
\else % if luatex or xelatex
  \ifxetex
    \usepackage{mathspec}
  \else
    \usepackage{fontspec}
  \fi
  \defaultfontfeatures{Ligatures=TeX,Scale=MatchLowercase}
\fi
% use upquote if available, for straight quotes in verbatim environments
\IfFileExists{upquote.sty}{\usepackage{upquote}}{}
% use microtype if available
\IfFileExists{microtype.sty}{%
\usepackage{microtype}
\UseMicrotypeSet[protrusion]{basicmath} % disable protrusion for tt fonts
}{}
\usepackage[margin=1in]{geometry}
\usepackage{hyperref}
\hypersetup{unicode=true,
            pdftitle={Implementation of EL model and academic proficiency},
            pdfauthor={Joe Ciesielski},
            pdfborder={0 0 0},
            breaklinks=true}
\urlstyle{same}  % don't use monospace font for urls
\usepackage{graphicx,grffile}
\makeatletter
\def\maxwidth{\ifdim\Gin@nat@width>\linewidth\linewidth\else\Gin@nat@width\fi}
\def\maxheight{\ifdim\Gin@nat@height>\textheight\textheight\else\Gin@nat@height\fi}
\makeatother
% Scale images if necessary, so that they will not overflow the page
% margins by default, and it is still possible to overwrite the defaults
% using explicit options in \includegraphics[width, height, ...]{}
\setkeys{Gin}{width=\maxwidth,height=\maxheight,keepaspectratio}
\IfFileExists{parskip.sty}{%
\usepackage{parskip}
}{% else
\setlength{\parindent}{0pt}
\setlength{\parskip}{6pt plus 2pt minus 1pt}
}
\setlength{\emergencystretch}{3em}  % prevent overfull lines
\providecommand{\tightlist}{%
  \setlength{\itemsep}{0pt}\setlength{\parskip}{0pt}}
\setcounter{secnumdepth}{0}
% Redefines (sub)paragraphs to behave more like sections
\ifx\paragraph\undefined\else
\let\oldparagraph\paragraph
\renewcommand{\paragraph}[1]{\oldparagraph{#1}\mbox{}}
\fi
\ifx\subparagraph\undefined\else
\let\oldsubparagraph\subparagraph
\renewcommand{\subparagraph}[1]{\oldsubparagraph{#1}\mbox{}}
\fi

%%% Use protect on footnotes to avoid problems with footnotes in titles
\let\rmarkdownfootnote\footnote%
\def\footnote{\protect\rmarkdownfootnote}

%%% Change title format to be more compact
\usepackage{titling}

% Create subtitle command for use in maketitle
\newcommand{\subtitle}[1]{
  \posttitle{
    \begin{center}\large#1\end{center}
    }
}

\setlength{\droptitle}{-2em}

  \title{Implementation of EL model and academic proficiency}
    \pretitle{\vspace{\droptitle}\centering\huge}
  \posttitle{\par}
    \author{Joe Ciesielski}
    \preauthor{\centering\large\emph}
  \postauthor{\par}
      \predate{\centering\large\emph}
  \postdate{\par}
    \date{9/12/2018}

\usepackage{array}
\usepackage{caption}
\usepackage{graphicx}
\usepackage{siunitx}
\usepackage[table]{xcolor}
\usepackage{multirow}
\usepackage{hhline}
\usepackage{calc}
\usepackage{tabularx}

\begin{document}
\maketitle

\hypertarget{analyzing-the-data}{%
\subsection{Analyzing the data}\label{analyzing-the-data}}

Before digging in to the data, I'd like to spend time with stakeholders
to understand the types of decisions that are being made with this
information. Ultimately the goals is to help inform those decisions and
I'd want my analysis to be able to intersect with those decision-making
processes.

Given my limited experience with the EL model, I'd first start with
exploring the relationship between the IR score and measures of academic
proficiency. Below is a correlation matrix of IR and school / district
proficiency variables.

I created two additional variables to add to this analysis. First is
``ir\_diff'', which is the precent difference between the IR score and
the IR target for each school. The second is ``school\_dist\_diff'',
which represents the precent difference between the school and district
proficiency for each subject.

\begin{center}\includegraphics{ciesielski_joseph_el_task_files/figure-latex/unnamed-chunk-1-1} \end{center}

\hypertarget{initial-findings}{%
\subsection{Initial findings}\label{initial-findings}}

A number of things stick out to me. First is that there is a strong
negative relationship between IR differential (the difference between IR
score and target) and IR target. This suggests that school with high IR
scores get closer to their targets (have a lower differential).

There also appears to be a week correlation betwen the IR score and
school proficiency. That suggests that IR score is not a direct measure
of school proficiency. IR score and school profiency independence means
that the IR rating is not just a proxy for academic success, which is a
positive finding.

However, the finding that stands out to me the most is the somewhat
strong negative relationship between IR differential and school /
district proficiency (outlined in the yellow box above). This means that
scores with higher rates are proficiency (or schools in districts with
higher rates of profieicny) are scoring closer to their targets while
schools with more limited academic success are more likley to score high
above their target.

\hypertarget{deeper-look-at-ir-differential-and-proficiency}{%
\subsection{Deeper look at IR differential and
proficiency}\label{deeper-look-at-ir-differential-and-proficiency}}

The first question I have about this finding is how the target score
imapcts the IR difference.

\begin{center}\includegraphics{ciesielski_joseph_el_task_files/figure-latex/unnamed-chunk-2-1} \end{center}

The chart obove shows the relationship between school proficiency and IR
difference for each of the three target scores. While schools with
higher targets targets score closer to those targets, the negative
relationship between proficiency and IR difference persists in for two
of the three target groups. Only two schools had a 66 target, so it's
difficult to say if this pattern holds true for those schools.

Since we only have one IR score for each school, but data on multiple
years and subjects for academic proficiency, we should also breakdown
the IR difference / proficiency relationship by some of these other
factors to see if our pattern persists.

\begin{center}\includegraphics{ciesielski_joseph_el_task_files/figure-latex/unnamed-chunk-3-1} \end{center}

Above we see that the IR difference decreases with increase school
proficiency for each year.

\begin{center}\includegraphics{ciesielski_joseph_el_task_files/figure-latex/unnamed-chunk-4-1} \end{center}

The chart above shows that IR difference decreases with school profiency
for both math and ELA.

\begin{center}\includegraphics{ciesielski_joseph_el_task_files/figure-latex/unnamed-chunk-5-1} \end{center}

Finaly, we see that the pattern holds for both charter and district
schools.

It's also useful to construct a model to determine the numerical
relationship between these variables. I created a simple linear
regression that compares IR difference to school proficiency accounting
for both year and subject.

\begin{table}[h]
\centering\captionsetup{justification=centering,singlelinecheck=off}
\caption{Regression table of IR difference}
\begin{tabularx}{0.5\textwidth}{p{0.1\textwidth} p{0.1\textwidth} p{0.1\textwidth} p{0.1\textwidth} p{0.1\textwidth}}


\hhline{}
\arrayrulecolor{black}

\multicolumn{1}{!{\vrule width 0pt}l!{\vrule width 0pt}}{\cellcolor[RGB]{255, 255, 255}\rule{0pt}{\baselineskip+4pt}\raggedright term\rule[-4pt]{0pt}{4pt}} &
\multicolumn{1}{l!{\vrule width 0pt}}{\cellcolor[RGB]{255, 255, 255}\rule{0pt}{\baselineskip+4pt}\raggedright estimate\rule[-4pt]{0pt}{4pt}} &
\multicolumn{1}{l!{\vrule width 0pt}}{\cellcolor[RGB]{255, 255, 255}\rule{0pt}{\baselineskip+4pt}\raggedright std.error\rule[-4pt]{0pt}{4pt}} &
\multicolumn{1}{l!{\vrule width 0pt}}{\cellcolor[RGB]{255, 255, 255}\rule{0pt}{\baselineskip+4pt}\raggedright statistic\rule[-4pt]{0pt}{4pt}} &
\multicolumn{1}{l!{\vrule width 0pt}}{\cellcolor[RGB]{255, 255, 255}\rule{0pt}{\baselineskip+4pt}\raggedright p.value\rule[-4pt]{0pt}{4pt}} \tabularnewline[-0.5pt]


\hhline{}
\arrayrulecolor{black}

\multicolumn{1}{!{\vrule width 0pt}l!{\vrule width 0pt}}{\cellcolor[RGB]{255, 255, 255}\rule{0pt}{\baselineskip+4pt}\raggedright (Intercept)\rule[-4pt]{0pt}{4pt}} &
\multicolumn{1}{r!{\vrule width 0pt}}{\cellcolor[RGB]{255, 255, 255}\rule{0pt}{\baselineskip+4pt}\raggedleft 0.237~\rule[-4pt]{0pt}{4pt}} &
\multicolumn{1}{r!{\vrule width 0pt}}{\cellcolor[RGB]{255, 255, 255}\rule{0pt}{\baselineskip+4pt}\raggedleft 0.0443\rule[-4pt]{0pt}{4pt}} &
\multicolumn{1}{r!{\vrule width 0pt}}{\cellcolor[RGB]{255, 255, 255}\rule{0pt}{\baselineskip+4pt}\raggedleft 5.35~\rule[-4pt]{0pt}{4pt}} &
\multicolumn{1}{r!{\vrule width 0pt}}{\cellcolor[RGB]{255, 255, 255}\rule{0pt}{\baselineskip+4pt}\raggedleft 7.42e-07\rule[-4pt]{0pt}{4pt}} \tabularnewline[-0.5pt]


\hhline{}
\arrayrulecolor{black}

\multicolumn{1}{!{\vrule width 0pt}l!{\vrule width 0pt}}{\cellcolor[RGB]{255, 255, 255}\rule{0pt}{\baselineskip+4pt}\raggedright school\_prof\rule[-4pt]{0pt}{4pt}} &
\multicolumn{1}{r!{\vrule width 0pt}}{\cellcolor[RGB]{255, 255, 255}\rule{0pt}{\baselineskip+4pt}\raggedleft -0.186~\rule[-4pt]{0pt}{4pt}} &
\multicolumn{1}{r!{\vrule width 0pt}}{\cellcolor[RGB]{255, 255, 255}\rule{0pt}{\baselineskip+4pt}\raggedleft 0.0489\rule[-4pt]{0pt}{4pt}} &
\multicolumn{1}{r!{\vrule width 0pt}}{\cellcolor[RGB]{255, 255, 255}\rule{0pt}{\baselineskip+4pt}\raggedleft -3.82~\rule[-4pt]{0pt}{4pt}} &
\multicolumn{1}{r!{\vrule width 0pt}}{\cellcolor[RGB]{255, 255, 255}\rule{0pt}{\baselineskip+4pt}\raggedleft 0.000256\rule[-4pt]{0pt}{4pt}} \tabularnewline[-0.5pt]


\hhline{}
\arrayrulecolor{black}

\multicolumn{1}{!{\vrule width 0pt}l!{\vrule width 0pt}}{\cellcolor[RGB]{255, 255, 255}\rule{0pt}{\baselineskip+4pt}\raggedright year2012-2013\rule[-4pt]{0pt}{4pt}} &
\multicolumn{1}{r!{\vrule width 0pt}}{\cellcolor[RGB]{255, 255, 255}\rule{0pt}{\baselineskip+4pt}\raggedleft -0.0221\rule[-4pt]{0pt}{4pt}} &
\multicolumn{1}{r!{\vrule width 0pt}}{\cellcolor[RGB]{255, 255, 255}\rule{0pt}{\baselineskip+4pt}\raggedleft 0.0233\rule[-4pt]{0pt}{4pt}} &
\multicolumn{1}{r!{\vrule width 0pt}}{\cellcolor[RGB]{255, 255, 255}\rule{0pt}{\baselineskip+4pt}\raggedleft -0.947\rule[-4pt]{0pt}{4pt}} &
\multicolumn{1}{r!{\vrule width 0pt}}{\cellcolor[RGB]{255, 255, 255}\rule{0pt}{\baselineskip+4pt}\raggedleft 0.346~~~\rule[-4pt]{0pt}{4pt}} \tabularnewline[-0.5pt]


\hhline{}
\arrayrulecolor{black}

\multicolumn{1}{!{\vrule width 0pt}l!{\vrule width 0pt}}{\cellcolor[RGB]{255, 255, 255}\rule{0pt}{\baselineskip+4pt}\raggedright year2013-2014\rule[-4pt]{0pt}{4pt}} &
\multicolumn{1}{r!{\vrule width 0pt}}{\cellcolor[RGB]{255, 255, 255}\rule{0pt}{\baselineskip+4pt}\raggedleft -0.0215\rule[-4pt]{0pt}{4pt}} &
\multicolumn{1}{r!{\vrule width 0pt}}{\cellcolor[RGB]{255, 255, 255}\rule{0pt}{\baselineskip+4pt}\raggedleft 0.0233\rule[-4pt]{0pt}{4pt}} &
\multicolumn{1}{r!{\vrule width 0pt}}{\cellcolor[RGB]{255, 255, 255}\rule{0pt}{\baselineskip+4pt}\raggedleft -0.924\rule[-4pt]{0pt}{4pt}} &
\multicolumn{1}{r!{\vrule width 0pt}}{\cellcolor[RGB]{255, 255, 255}\rule{0pt}{\baselineskip+4pt}\raggedleft 0.358~~~\rule[-4pt]{0pt}{4pt}} \tabularnewline[-0.5pt]


\hhline{}
\arrayrulecolor{black}

\multicolumn{1}{!{\vrule width 0pt}l!{\vrule width 0pt}}{\cellcolor[RGB]{255, 255, 255}\rule{0pt}{\baselineskip+4pt}\raggedright subjectmath\rule[-4pt]{0pt}{4pt}} &
\multicolumn{1}{r!{\vrule width 0pt}}{\cellcolor[RGB]{255, 255, 255}\rule{0pt}{\baselineskip+4pt}\raggedleft -0.0225\rule[-4pt]{0pt}{4pt}} &
\multicolumn{1}{r!{\vrule width 0pt}}{\cellcolor[RGB]{255, 255, 255}\rule{0pt}{\baselineskip+4pt}\raggedleft 0.0194\rule[-4pt]{0pt}{4pt}} &
\multicolumn{1}{r!{\vrule width 0pt}}{\cellcolor[RGB]{255, 255, 255}\rule{0pt}{\baselineskip+4pt}\raggedleft -1.16~\rule[-4pt]{0pt}{4pt}} &
\multicolumn{1}{r!{\vrule width 0pt}}{\cellcolor[RGB]{255, 255, 255}\rule{0pt}{\baselineskip+4pt}\raggedleft 0.248~~~\rule[-4pt]{0pt}{4pt}} \tabularnewline[-0.5pt]


\hhline{}
\arrayrulecolor{black}
\end{tabularx}
\end{table}

The table above says that if a school goes from 0\% to 100\% proficient,
regardless of year or subject, we can expect about a 19 point decrease
in the precent difference between IR score and IR target. Obviously, a
school with 0\% proficient is not realistic, but this does help to
quantify the strength of the relationship.

One possible explanation for this finding is that there are ceiling
effects. This is the idea that there is some maximum score and that as
you get closer to this score, large changes become more challening. It
may be worth exploring how targets are set to ensure that schools with
high rates of academic proficiency have targets set which are
challenging for them.

\hypertarget{sharing-findings}{%
\subsection{Sharing findings}\label{sharing-findings}}

I would like to share the above finding with stakeholders, specifically
to explore individual schools to determine if it holds true and help to
inform decisions.

\begin{center}\includegraphics{ciesielski_joseph_el_task_files/figure-latex/unnamed-chunk-7-1} \end{center}

The chart above shows the residuals for each school. The residuals are
the predicted difference between IR score and IR target based on school
proficiency rates. High residuals show scores that are greatly
outperforming their predicited difference; low residuals show schools
that are at or below their predicted difference.

School 2 seems to be exceeding their target by a large amount. Below is
some specific data about this school.

\begin{table}[h]
\centering\begin{tabularx}{0.5\textwidth}{p{0.25\textwidth} p{0.25\textwidth}}


\hhline{}
\arrayrulecolor{black}

\multicolumn{1}{!{\vrule width 0pt}l!{\vrule width 0pt}}{\cellcolor[RGB]{255, 255, 255}\rule{0pt}{\baselineskip+4pt}\raggedright school\_name\rule[-4pt]{0pt}{4pt}} &
\multicolumn{1}{l!{\vrule width 0pt}}{\cellcolor[RGB]{255, 255, 255}\rule{0pt}{\baselineskip+4pt}\raggedright School 2\rule[-4pt]{0pt}{4pt}} \tabularnewline[-0.5pt]


\hhline{}
\arrayrulecolor{black}

\multicolumn{1}{!{\vrule width 0pt}l!{\vrule width 0pt}}{\cellcolor[RGB]{255, 255, 255}\rule{0pt}{\baselineskip+4pt}\raggedright governance\rule[-4pt]{0pt}{4pt}} &
\multicolumn{1}{l!{\vrule width 0pt}}{\cellcolor[RGB]{255, 255, 255}\rule{0pt}{\baselineskip+4pt}\raggedright Charter\rule[-4pt]{0pt}{4pt}} \tabularnewline[-0.5pt]


\hhline{}
\arrayrulecolor{black}

\multicolumn{1}{!{\vrule width 0pt}l!{\vrule width 0pt}}{\cellcolor[RGB]{255, 255, 255}\rule{0pt}{\baselineskip+4pt}\raggedright setting\rule[-4pt]{0pt}{4pt}} &
\multicolumn{1}{l!{\vrule width 0pt}}{\cellcolor[RGB]{255, 255, 255}\rule{0pt}{\baselineskip+4pt}\raggedright Suburban\rule[-4pt]{0pt}{4pt}} \tabularnewline[-0.5pt]


\hhline{}
\arrayrulecolor{black}

\multicolumn{1}{!{\vrule width 0pt}l!{\vrule width 0pt}}{\cellcolor[RGB]{255, 255, 255}\rule{0pt}{\baselineskip+4pt}\raggedright \_subgroup\_ell\rule[-4pt]{0pt}{4pt}} &
\multicolumn{1}{l!{\vrule width 0pt}}{\cellcolor[RGB]{255, 255, 255}\rule{0pt}{\baselineskip+4pt}\raggedright 0.106\rule[-4pt]{0pt}{4pt}} \tabularnewline[-0.5pt]


\hhline{}
\arrayrulecolor{black}

\multicolumn{1}{!{\vrule width 0pt}l!{\vrule width 0pt}}{\cellcolor[RGB]{255, 255, 255}\rule{0pt}{\baselineskip+4pt}\raggedright \_subgroup\_frl\rule[-4pt]{0pt}{4pt}} &
\multicolumn{1}{l!{\vrule width 0pt}}{\cellcolor[RGB]{255, 255, 255}\rule{0pt}{\baselineskip+4pt}\raggedright 0.781\rule[-4pt]{0pt}{4pt}} \tabularnewline[-0.5pt]


\hhline{}
\arrayrulecolor{black}

\multicolumn{1}{!{\vrule width 0pt}l!{\vrule width 0pt}}{\cellcolor[RGB]{255, 255, 255}\rule{0pt}{\baselineskip+4pt}\raggedright years\_as\_el\_school\rule[-4pt]{0pt}{4pt}} &
\multicolumn{1}{l!{\vrule width 0pt}}{\cellcolor[RGB]{255, 255, 255}\rule{0pt}{\baselineskip+4pt}\raggedright 2\rule[-4pt]{0pt}{4pt}} \tabularnewline[-0.5pt]


\hhline{}
\arrayrulecolor{black}

\multicolumn{1}{!{\vrule width 0pt}l!{\vrule width 0pt}}{\cellcolor[RGB]{255, 255, 255}\rule{0pt}{\baselineskip+4pt}\raggedright ir\_score\rule[-4pt]{0pt}{4pt}} &
\multicolumn{1}{l!{\vrule width 0pt}}{\cellcolor[RGB]{255, 255, 255}\rule{0pt}{\baselineskip+4pt}\raggedright 88\rule[-4pt]{0pt}{4pt}} \tabularnewline[-0.5pt]


\hhline{}
\arrayrulecolor{black}

\multicolumn{1}{!{\vrule width 0pt}l!{\vrule width 0pt}}{\cellcolor[RGB]{255, 255, 255}\rule{0pt}{\baselineskip+4pt}\raggedright ir\_target\rule[-4pt]{0pt}{4pt}} &
\multicolumn{1}{l!{\vrule width 0pt}}{\cellcolor[RGB]{255, 255, 255}\rule{0pt}{\baselineskip+4pt}\raggedright 66\rule[-4pt]{0pt}{4pt}} \tabularnewline[-0.5pt]


\hhline{}
\arrayrulecolor{black}

\multicolumn{1}{!{\vrule width 0pt}l!{\vrule width 0pt}}{\cellcolor[RGB]{255, 255, 255}\rule{0pt}{\baselineskip+4pt}\raggedright ela 2011-2012\rule[-4pt]{0pt}{4pt}} &
\multicolumn{1}{l!{\vrule width 0pt}}{\cellcolor[RGB]{255, 255, 255}\rule{0pt}{\baselineskip+4pt}\raggedright 0.573\rule[-4pt]{0pt}{4pt}} \tabularnewline[-0.5pt]


\hhline{}
\arrayrulecolor{black}

\multicolumn{1}{!{\vrule width 0pt}l!{\vrule width 0pt}}{\cellcolor[RGB]{255, 255, 255}\rule{0pt}{\baselineskip+4pt}\raggedright ela 2012-2013\rule[-4pt]{0pt}{4pt}} &
\multicolumn{1}{l!{\vrule width 0pt}}{\cellcolor[RGB]{255, 255, 255}\rule{0pt}{\baselineskip+4pt}\raggedright 0.608\rule[-4pt]{0pt}{4pt}} \tabularnewline[-0.5pt]


\hhline{}
\arrayrulecolor{black}

\multicolumn{1}{!{\vrule width 0pt}l!{\vrule width 0pt}}{\cellcolor[RGB]{255, 255, 255}\rule{0pt}{\baselineskip+4pt}\raggedright ela 2013-2014\rule[-4pt]{0pt}{4pt}} &
\multicolumn{1}{l!{\vrule width 0pt}}{\cellcolor[RGB]{255, 255, 255}\rule{0pt}{\baselineskip+4pt}\raggedright 0.519\rule[-4pt]{0pt}{4pt}} \tabularnewline[-0.5pt]


\hhline{}
\arrayrulecolor{black}

\multicolumn{1}{!{\vrule width 0pt}l!{\vrule width 0pt}}{\cellcolor[RGB]{255, 255, 255}\rule{0pt}{\baselineskip+4pt}\raggedright math 2011-2012\rule[-4pt]{0pt}{4pt}} &
\multicolumn{1}{l!{\vrule width 0pt}}{\cellcolor[RGB]{255, 255, 255}\rule{0pt}{\baselineskip+4pt}\raggedright 0.546\rule[-4pt]{0pt}{4pt}} \tabularnewline[-0.5pt]


\hhline{}
\arrayrulecolor{black}

\multicolumn{1}{!{\vrule width 0pt}l!{\vrule width 0pt}}{\cellcolor[RGB]{255, 255, 255}\rule{0pt}{\baselineskip+4pt}\raggedright math 2012-2013\rule[-4pt]{0pt}{4pt}} &
\multicolumn{1}{l!{\vrule width 0pt}}{\cellcolor[RGB]{255, 255, 255}\rule{0pt}{\baselineskip+4pt}\raggedright 0.511\rule[-4pt]{0pt}{4pt}} \tabularnewline[-0.5pt]


\hhline{}
\arrayrulecolor{black}

\multicolumn{1}{!{\vrule width 0pt}l!{\vrule width 0pt}}{\cellcolor[RGB]{255, 255, 255}\rule{0pt}{\baselineskip+4pt}\raggedright math 2013-2014\rule[-4pt]{0pt}{4pt}} &
\multicolumn{1}{l!{\vrule width 0pt}}{\cellcolor[RGB]{255, 255, 255}\rule{0pt}{\baselineskip+4pt}\raggedright 0.514\rule[-4pt]{0pt}{4pt}} \tabularnewline[-0.5pt]


\hhline{}
\arrayrulecolor{black}
\end{tabularx}
\end{table}

School 2 has a high percentage of low-income students (based on free and
reduced lunch), has been an EL school for two years, has an IR target of
66, and conistently has slightly more than half of their students
proficient. They far out-paced their target by getting an 88 on the IR,
a greater difference than any other school. Assuming this target was
appropriate for them, it may be worth exploring what contributed to
their success, and using this school as a model for others.

\begin{table}[h]
\centering\begin{tabularx}{0.5\textwidth}{p{0.25\textwidth} p{0.25\textwidth}}


\hhline{}
\arrayrulecolor{black}

\multicolumn{1}{!{\vrule width 0pt}l!{\vrule width 0pt}}{\cellcolor[RGB]{255, 255, 255}\rule{0pt}{\baselineskip+4pt}\raggedright school\_name\rule[-4pt]{0pt}{4pt}} &
\multicolumn{1}{l!{\vrule width 0pt}}{\cellcolor[RGB]{255, 255, 255}\rule{0pt}{\baselineskip+4pt}\raggedright School 7\rule[-4pt]{0pt}{4pt}} \tabularnewline[-0.5pt]


\hhline{}
\arrayrulecolor{black}

\multicolumn{1}{!{\vrule width 0pt}l!{\vrule width 0pt}}{\cellcolor[RGB]{255, 255, 255}\rule{0pt}{\baselineskip+4pt}\raggedright governance\rule[-4pt]{0pt}{4pt}} &
\multicolumn{1}{l!{\vrule width 0pt}}{\cellcolor[RGB]{255, 255, 255}\rule{0pt}{\baselineskip+4pt}\raggedright District\rule[-4pt]{0pt}{4pt}} \tabularnewline[-0.5pt]


\hhline{}
\arrayrulecolor{black}

\multicolumn{1}{!{\vrule width 0pt}l!{\vrule width 0pt}}{\cellcolor[RGB]{255, 255, 255}\rule{0pt}{\baselineskip+4pt}\raggedright setting\rule[-4pt]{0pt}{4pt}} &
\multicolumn{1}{l!{\vrule width 0pt}}{\cellcolor[RGB]{255, 255, 255}\rule{0pt}{\baselineskip+4pt}\raggedright Urban\rule[-4pt]{0pt}{4pt}} \tabularnewline[-0.5pt]


\hhline{}
\arrayrulecolor{black}

\multicolumn{1}{!{\vrule width 0pt}l!{\vrule width 0pt}}{\cellcolor[RGB]{255, 255, 255}\rule{0pt}{\baselineskip+4pt}\raggedright \_subgroup\_ell\rule[-4pt]{0pt}{4pt}} &
\multicolumn{1}{l!{\vrule width 0pt}}{\cellcolor[RGB]{255, 255, 255}\rule{0pt}{\baselineskip+4pt}\raggedright 0.221\rule[-4pt]{0pt}{4pt}} \tabularnewline[-0.5pt]


\hhline{}
\arrayrulecolor{black}

\multicolumn{1}{!{\vrule width 0pt}l!{\vrule width 0pt}}{\cellcolor[RGB]{255, 255, 255}\rule{0pt}{\baselineskip+4pt}\raggedright \_subgroup\_frl\rule[-4pt]{0pt}{4pt}} &
\multicolumn{1}{l!{\vrule width 0pt}}{\cellcolor[RGB]{255, 255, 255}\rule{0pt}{\baselineskip+4pt}\raggedright 0.445\rule[-4pt]{0pt}{4pt}} \tabularnewline[-0.5pt]


\hhline{}
\arrayrulecolor{black}

\multicolumn{1}{!{\vrule width 0pt}l!{\vrule width 0pt}}{\cellcolor[RGB]{255, 255, 255}\rule{0pt}{\baselineskip+4pt}\raggedright years\_as\_el\_school\rule[-4pt]{0pt}{4pt}} &
\multicolumn{1}{l!{\vrule width 0pt}}{\cellcolor[RGB]{255, 255, 255}\rule{0pt}{\baselineskip+4pt}\raggedright 9\rule[-4pt]{0pt}{4pt}} \tabularnewline[-0.5pt]


\hhline{}
\arrayrulecolor{black}

\multicolumn{1}{!{\vrule width 0pt}l!{\vrule width 0pt}}{\cellcolor[RGB]{255, 255, 255}\rule{0pt}{\baselineskip+4pt}\raggedright ir\_score\rule[-4pt]{0pt}{4pt}} &
\multicolumn{1}{l!{\vrule width 0pt}}{\cellcolor[RGB]{255, 255, 255}\rule{0pt}{\baselineskip+4pt}\raggedright 97\rule[-4pt]{0pt}{4pt}} \tabularnewline[-0.5pt]


\hhline{}
\arrayrulecolor{black}

\multicolumn{1}{!{\vrule width 0pt}l!{\vrule width 0pt}}{\cellcolor[RGB]{255, 255, 255}\rule{0pt}{\baselineskip+4pt}\raggedright ir\_target\rule[-4pt]{0pt}{4pt}} &
\multicolumn{1}{l!{\vrule width 0pt}}{\cellcolor[RGB]{255, 255, 255}\rule{0pt}{\baselineskip+4pt}\raggedright 98\rule[-4pt]{0pt}{4pt}} \tabularnewline[-0.5pt]


\hhline{}
\arrayrulecolor{black}

\multicolumn{1}{!{\vrule width 0pt}l!{\vrule width 0pt}}{\cellcolor[RGB]{255, 255, 255}\rule{0pt}{\baselineskip+4pt}\raggedright ela 2011-2012\rule[-4pt]{0pt}{4pt}} &
\multicolumn{1}{l!{\vrule width 0pt}}{\cellcolor[RGB]{255, 255, 255}\rule{0pt}{\baselineskip+4pt}\raggedright 0.818\rule[-4pt]{0pt}{4pt}} \tabularnewline[-0.5pt]


\hhline{}
\arrayrulecolor{black}

\multicolumn{1}{!{\vrule width 0pt}l!{\vrule width 0pt}}{\cellcolor[RGB]{255, 255, 255}\rule{0pt}{\baselineskip+4pt}\raggedright ela 2012-2013\rule[-4pt]{0pt}{4pt}} &
\multicolumn{1}{l!{\vrule width 0pt}}{\cellcolor[RGB]{255, 255, 255}\rule{0pt}{\baselineskip+4pt}\raggedright 0.788\rule[-4pt]{0pt}{4pt}} \tabularnewline[-0.5pt]


\hhline{}
\arrayrulecolor{black}

\multicolumn{1}{!{\vrule width 0pt}l!{\vrule width 0pt}}{\cellcolor[RGB]{255, 255, 255}\rule{0pt}{\baselineskip+4pt}\raggedright ela 2013-2014\rule[-4pt]{0pt}{4pt}} &
\multicolumn{1}{l!{\vrule width 0pt}}{\cellcolor[RGB]{255, 255, 255}\rule{0pt}{\baselineskip+4pt}\raggedright 0.391\rule[-4pt]{0pt}{4pt}} \tabularnewline[-0.5pt]


\hhline{}
\arrayrulecolor{black}

\multicolumn{1}{!{\vrule width 0pt}l!{\vrule width 0pt}}{\cellcolor[RGB]{255, 255, 255}\rule{0pt}{\baselineskip+4pt}\raggedright math 2011-2012\rule[-4pt]{0pt}{4pt}} &
\multicolumn{1}{l!{\vrule width 0pt}}{\cellcolor[RGB]{255, 255, 255}\rule{0pt}{\baselineskip+4pt}\raggedright 0.605\rule[-4pt]{0pt}{4pt}} \tabularnewline[-0.5pt]


\hhline{}
\arrayrulecolor{black}

\multicolumn{1}{!{\vrule width 0pt}l!{\vrule width 0pt}}{\cellcolor[RGB]{255, 255, 255}\rule{0pt}{\baselineskip+4pt}\raggedright math 2012-2013\rule[-4pt]{0pt}{4pt}} &
\multicolumn{1}{l!{\vrule width 0pt}}{\cellcolor[RGB]{255, 255, 255}\rule{0pt}{\baselineskip+4pt}\raggedright 0.598\rule[-4pt]{0pt}{4pt}} \tabularnewline[-0.5pt]


\hhline{}
\arrayrulecolor{black}

\multicolumn{1}{!{\vrule width 0pt}l!{\vrule width 0pt}}{\cellcolor[RGB]{255, 255, 255}\rule{0pt}{\baselineskip+4pt}\raggedright math 2013-2014\rule[-4pt]{0pt}{4pt}} &
\multicolumn{1}{l!{\vrule width 0pt}}{\cellcolor[RGB]{255, 255, 255}\rule{0pt}{\baselineskip+4pt}\raggedright 0.289\rule[-4pt]{0pt}{4pt}} \tabularnewline[-0.5pt]


\hhline{}
\arrayrulecolor{black}
\end{tabularx}
\end{table}

School 7 had the lowest residuals. Being an EL school for 9 years, they
had an IR target of 98. They had high rates of profieincy for two years
of the data but saw major drops in proficiency in 2013-14. There is
likley some larger context here, such as changes in school leadership,
testing, or neighborhood, that is worth exploring.

The finding above becomes useful when we examine how it plays out in
individual schools and district. It would be important to work with EL
staff to determine how, if at all, this might impact their work with
schools.

In order to share the rest of the data with schools and districts, it
would I would create two resources. The first would be an ineractive
tool which would allow schools and districts to explore the data on
their own. This would allow them to modify visulizations so that they
could explore the realtionship between IR score and math proficiency,
for example. They could drill down to specific schools or pull back to
look at whole districts. I have found putting data in the hands of
practioners and stakeholders generates useufl insights and questions
which can then be explored with more in-depth analysis.

However, there is still the need to have some summative findings. For
each school, I would create a report which shared their final IR score
(ideally including measures for each of the core practices rather than
just a final number), alongside their school demographics and
proficiency rates. It would show how their school compares to similar
school. These could be used to facilitate conversations about areas of
strength and growth. District would receive similar reports that
included a district level version with information about each of their
schools.


\end{document}
